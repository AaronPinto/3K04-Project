\documentclass[12pt]{article}
\usepackage[letterpaper, margin=1in]{geometry}
\usepackage{titling} \predate{} \postdate{}
\usepackage{parskip}
\usepackage{multirow}
\usepackage{graphicx}
\usepackage{pifont}
\newcommand{\cmark}{\ding{51}}
\long\def\/*#1*/{}
\usepackage{hhline}
\usepackage{booktabs}

\title{SFWRENG 3K04 \\ Assignment 1 Part 2 \\ Documentation}
\author{
    Lab 1 Group 7 \\
    Carlos Capili \\
    Raeed Hassan \\
    Shaqeeb Momen \\
    Aaron Pinto \\
    Udeep Shah
}
\date{}

\setlength{\aboverulesep}{0pt}
\setlength{\belowrulesep}{0pt}

\begin{document}
\maketitle \newpage
\tableofcontents \newpage

\section{Requirements}

\subsection{Welcome Screen}
%Develop an interface that includes a welcome screen, including the ability to register a new user (name and password), and to login as an existing user. A maximum of 10 users should be allowed to be stored locally.
All required aspects of the welcome screen are implemented in the current software revision: 
\begin{itemize}
    \item Register a new user
    \item Login as an existing user
    \item Maximum of 10 users stored locally
\end{itemize}
All requirements will remain in future software revisions.  

\subsection{User Interface Essential Aspects}
All required essential aspects of the user interface have been implemented in the current software revision:
\begin{itemize}
    \item User interface is capable of utilizing and managing windows for display of text and graphics
    \item User interface is capable of processing user positioning and input buttons
    \item User interface is capable of displaying all programmable parameters for review and modification
    \item User interface is capable of visually indicating when the DCM and the device are communicating
    \item User interface is capable of visually indicating when a different PACEMAKER device is approached than was previously interrogated
\end{itemize}
All requirements will remain in future software revisions.

\subsection{DCM Utility Functions}
\subsubsection{About}
All required aspects of the About function are implemented in the current software revision: 
\begin{itemize}
    \item Application model number
    \item Application software revision
    \item DCM serial number
    \item Institution name
\end{itemize}
All requirements will remain in future software revisions.

\subsubsection{Set Clock}
The Set Clock function is not implemented, but has been allocated a window and a button on the DCM main window. This function will likely be removed in a future revision of the software depending on the feasibility of implementing the feature.

\subsubsection{New Patient}
The New Patient function is implemented in the current software revision. The function will remain in future software revisions.

\subsubsection{Quit}
The Quit function is implemented in the current software revision. The function will remain in future software revisions.

\subsection{Pacing Mode Interfaces}
Interfaces for pacing modes and pacing mode selection are implemented in the current software revision. The feature will remain in future software revisions. 

\subsection{Printed Reports}
All common requirements for printed reports are implemented in the current software revision:
\begin{itemize}
    \item Header Information
    \begin{itemize}
        \item Application model and version number
        \item Device model and serial number
        \item DCM serial number
        \item Date and time of report printing
        \item Report name 
    \end{itemize}
\end{itemize}
All common requirements for printed reports will remain in future software revisions.
\subsubsection{Bradycardia Parameters}
The feature is implemented in the current software revision. The feature will remain in future software revisions.
\subsubsection{Temporary Parameters}
The feature has been allocated a button in the reports window in the current software revision. This feature will likely be removed in a future revision of the software as it is unlikely that a temporary pacing mode is introduced.

\subsection{Programmable Parameters}
Provisions for storing programmable parameter data have been implemented in the current software revision for the following programmable paramters:
\begin{itemize}
    \item Lower Rate Limit
    \item Upper Rate Limit
    \item Atrial Amplitude
    \item Atrial Pulse Width
    \item Atrial Refractory Period (ARP)
    \item Ventricular Amplitude
    \item Ventricular Pulse Width
    \item Ventricular Refractory Period (VRP)
\end{itemize}
These programmable parameters will remain in future software revisions. 

\subsection{Real-time Electrograms}
stuff

\newpage
\section{Software}
\subsection{}

\newpage
\section{Testing}
\subsection{}

\end{document}